\chapter{e}\label{apendice:b}

\section{\_\_init\_\_.py}
\begin{minted}[fontsize={\fontsize{5.5}{6.5}\selectfont}, breaklines]{python}
import numpy as np
from .src.lib.matplotlib_settings import plt
import pandas as pd
\end{minted}


\section{src}

\subsection{core}

\subsubsection{\_\_init\_\_.py}
\begin{minted}[fontsize={\fontsize{5.5}{6.5}\selectfont}, breaklines]{python}

\end{minted}



\subsubsection{\_abstractas.py}
\begin{minted}[fontsize={\fontsize{5.5}{6.5}\selectfont}, breaklines]{python}
# /e/src/core/_abstractas.py

from abc import ABC, abstractmethod
from ._typing import (
    ListaStringsLike,
    DictParametrosLike
)


class Parametros(ABC):
    
    """
    # Explicacion
    Esta clase pretende facilitar el uso de guardado de parametros.
    
    ## Example
    >>> class SDEModelParameters(Parameters):
    >>>     mu = 2
    >>>     sigma = 1
    >>>     X0 = 1
    >>> 
    >>> params = SDEModelParameters()
    >>> print(params)  # Salida esperada: "Parameters: SDEModelParameters"
    >>> print(params.nombre) # Salida esperada: "SDEModelParameters"
    >>> print(params.parametros_de_la_clase())  # {'mu': 2, 'sigma': 1, 'X0': 1}
    """
    
    def __init__(self, **kwargs):
        for key, value in kwargs.items():
            setattr(self, key, value)

    def __repr__(self) -> str:
        base_class_name = self.__class__.__bases__[0].__name__
        return f"{base_class_name}: {self.__class__.__name__}"

    @property
    def nombre(self) -> str:
        return self.__repr__().split(':')[1].strip()

    @classmethod
    def lista_de_funciones_prop_de_una_clase(cls) -> ListaStringsLike:
        return [p for p in dir(cls) if isinstance(getattr(cls, p), property)]

    @classmethod
    def parametros_de_la_clase(cls) -> DictParametrosLike:
        # Obtener las propiedades de la clase
        property_names = cls.lista_de_funciones_prop_de_una_clase()
        
        # Obtener todos los atributos que no sean métodos ni propiedades internas
        internal_variables_dict = {k: v for k, v in vars(cls).items() if not k.startswith("__")}
        
        # Excluir las propiedades y los atributos internos que comienzan con "_"
        store_keys = [] + property_names
        for key in internal_variables_dict.keys():
            if key.startswith('_'): 
                store_keys.append(key)
        for key in store_keys:
            if key in internal_variables_dict:
                del internal_variables_dict[key]
        
        return internal_variables_dict
\end{minted}


\subsubsection{\_typing.py}
\begin{minted}[fontsize={\fontsize{5.5}{6.5}\selectfont}, breaklines]{python}
# /e/src/core/_typing.py

from ... import np

from typing import (
    Dict, 
    Any,
    Union,
    Tuple,
    List,
    Optional,
    TypeVar,
    Generic,
    Callable
)

__all__ = [
    'Dict', 
    'Any',
    'Union',
    'Tuple',
    'List',
    'Optional',
    'TypeVar',
    'Generic',
    'Callable',
    'HiperparametrosLike',
    'ModuloLike',
    'CoordenadaLike',
    'CoordenadasLike',
    'DictOptionsLIke',
    'InputLike',
    'ResultadosLike',
    'ListaStringsLike',
    'DictParametrosLike',
    'ListaIntLike'
]

type ResultadosLike = Dict[str, Any]
type DictOptionsLIke = Dict[str, Any]
type DictParametrosLike = Dict[str, Any]
type HiperparametrosLike = Dict[str, int | float]
type ModuloLike = float
type CoordenadaLike = float | np.ndarray | int
type CoordenadasLike = Tuple[CoordenadaLike, ...]
type InputLike = Tuple[int | float]
type ListaStringsLike = List[str]
type ListaIntLike = List[int]

\end{minted}


\subsection{lib}

\subsubsection{\_\_init\_\_.py}
\begin{minted}[fontsize={\fontsize{5.5}{6.5}\selectfont}, breaklines]{python}

\end{minted}


\subsubsection{constants.py}
\begin{minted}[fontsize={\fontsize{5.5}{6.5}\selectfont}, breaklines]{python}
# /e/src/lib/constants.py

import os
import sys
from pathlib import Path
from dataclasses import dataclass

__all__ = [
    'Rutas'
]

@dataclass(frozen=True)
class Rutas:
    RUTA_PAQUETE: str = str(Path(__file__).resolve().parents[2])
\end{minted}


\subsubsection{general.py}
\begin{minted}[fontsize={\fontsize{5.5}{6.5}\selectfont}, breaklines]{python}
# /e/lib/clases.py

from ..core._abstractas import Parametros
from ..core._typing import (
    InputsLike
)

class ParametrosProblema(Parametros):
    
    """
    Ejemplo
    ---
    >>> Temperatura.T0 = 0
    >>> Temperatura.T1 = 50
    >>> NpuntosDireccion.Nx = 100
    >>> NpuntosDireccion.Ny = 100
    >>> SemiCirculoParametros.R = 1

    >>> inputs = {
    >>>     'T0' : Temperatura.T0,
    >>>     'T1' : Temperatura.T1,
    >>>     'Nx' : NpuntosDireccion.Nx,
    >>>     'Ny' : NpuntosDireccion.Ny,
    >>>     'R' : SemiCirculoParametros.R
    >>> }

    >>> params = ParametrosProblema(dict_parametros=inputs)
    >>> params.print_parametros
    """
    
    def __init__(self, dict_parametros: InputsLike) -> None:
        self.inputs = dict_parametros
        
    def __repr__(self) -> str:
        return f"ParametrosProblema({list(self.inputs.keys())})"
        
    @property
    def print_parametros(self):
        for parametro, valor in self.inputs.items():
            print(f"{parametro:3} | {valor:6}")   
\end{minted}


\subsubsection{logger.py}
\begin{minted}[fontsize={\fontsize{5.5}{6.5}\selectfont}, breaklines]{python}
# /e/src/lib/logger.py

import logging
from logging import _nameToLevel, _levelToName
from e.src.core._typing import (
    DictParametrosLike,
)

__all__ = [
    'dict_log_level',
    'dict_level_log',
    'define_logger'
]

dict_log_level: DictParametrosLike = _nameToLevel
dict_level_log: DictParametrosLike = _levelToName

def define_logger(logger_name: str, logger_level: str = 'DEBUG'):
    
    # Configurar logger.
    logger = logging.getLogger(logger_name)
    logger.setLevel(dict_log_level[logger_level])

    console_handler = logging.StreamHandler()

    # Añadir un formato básico para los mensajes de log.
    formatter = logging.Formatter('%(name)s - %(levelname)s - %(message)s')
    console_handler.setFormatter(formatter)

    # Añadir el handler al logger.
    logger.addHandler(console_handler)
    
    return logger
\end{minted}


\subsubsection{matplotlib\_settings.py}
\begin{minted}[fontsize={\fontsize{5.5}{6.5}\selectfont}, breaklines]{python}
# /e/src/lib/matplotlib_settings.py

import matplotlib.pyplot as plt

plt.rcParams['figure.figsize'] = (9,6)
plt.rcParams['lines.linewidth'] = 3
plt.rcParams['xtick.bottom'] = False
plt.rcParams['ytick.left'] = False
pal = ["#FBB4AE","#B3CDE3", "#CCEBC5","#CFCCC4"]
\end{minted}


\subsubsection{metodos\_numericos.py}
\begin{minted}[fontsize={\fontsize{5.5}{6.5}\selectfont}, breaklines]{python}
# /e/src/lib/metodos_numericos.py

from ... import np
from .logger import define_logger
from ..core._typing import Callable

__all__ = [
    'solve_wave_eq',
    'jacobi',
    'gauss_seidel',
    'gauss_seidel_sor'
]

informer = define_logger(logger_name='mna', logger_level='INFO')

def DiferenciasFinitas2D(
    u: np.ndarray, 
    Nx: int, 
    Ny: int, 
    update_rule: Callable[[np.ndarray, int, int], float],  # Función para actualizar u[i,j]
    tol: float = 1e-6,
    max_iter: int = int(1e4)
    ) -> np.ndarray:
    
    for iteracion in range(max_iter):
        u_old = np.copy(u)
        
        # Iterar sobre los puntos internos.
        for i in range(1, Nx-1):
            for j in range(1, Ny-1):
                # Llamada a la regla de actualización que depende de la ecuación.
                u[i, j] = update_rule(u_old, i, j)
        
        # Criterio de convergencia
        error = np.max(np.abs(u - u_old))
        if error < tol: 
            print(f"Convergencia alcanzada después de {iteracion} iteraciones.")
            return u, iteracion
        
    print(f"No se alcanzó la convergencia después de {max_iter} iteraciones.")
    return u, max_iter

def solve_wave_eq(Nx, Nt, L, T, cfl):
    """
    Resuelve la ecuación de onda hiperbólica con el esquema explícito en diferencias finitas.
    
    Args:
    - Nx: Número de puntos en la dirección espacial (x).
    - Nt: Número de puntos en la dirección temporal (t).
    - L: Longitud del dominio espacial.
    - T: Tiempo total a simular.
    - cfl: Número de Courant (CFL), define la relación entre dt y dx.
    
    Returns:
    - u: Matriz con las soluciones aproximadas.
    - x: Vector de posiciones espaciales.
    - t: Vector de tiempos.
    """
    # Discretización espacial y temporal
    dx = L / (Nx - 1)
    dt = cfl * dx  # Para mantener la estabilidad, dt <= dx/c
    x = np.linspace(0, L, Nx)
    t = np.linspace(0, T, Nt)
    
    # Coeficiente de estabilidad CFL
    r = (dt / dx)**2
    
    # Inicialización de la solución
    u = np.zeros((Nt, Nx))
    informer.debug(u)
    
    # Condiciones iniciales
    u[0, :] = x * (1 - x)  # u(x, 0) = x(1 - x)
    
    # Primera iteración: derivada temporal cero
    u[1, :] = u[0, :]  # u_t(x, 0) = 0 implica que u[1, :] = u[0, :]
    
    # Aplicar condiciones de frontera
    u[:, 0] = 0  # u(0, t) = 0
    u[:, -1] = 0  # u(1, t) = 0
    
    # Iteraciones en el tiempo
    for n in range(1, Nt-1):
        for i in range(1, Nx-1):
            u[n+1, i] = (2 * u[n, i] - u[n-1, i] +
                         r * (u[n, i+1] - 2 * u[n, i] + u[n, i-1]) +
                         dt**2 * (1 - x[i]**2))
    
    return u, x, t


def solve_wave_eq(Nx, Nt, L, T, cfl):
    """
    Resuelve la ecuación de onda hiperbólica con el esquema explícito en diferencias finitas.
    
    Args:
    - Nx: Número de puntos en la dirección espacial (x).
    - Nt: Número de puntos en la dirección temporal (t).
    - L: Longitud del dominio espacial.
    - T: Tiempo total a simular.
    - cfl: Número de Courant (CFL), define la relación entre dt y dx.
    
    Returns:
    - u: Matriz con las soluciones aproximadas.
    - x: Vector de posiciones espaciales.
    - t: Vector de tiempos.
    """
    # Discretización espacial y temporal
    dx = L / (Nx - 1)
    dt = cfl * dx  # Para mantener la estabilidad, dt <= dx/c
    x = np.linspace(0, L, Nx)
    t = np.linspace(0, T, Nt)
    
    # Coeficiente de estabilidad CFL
    r = (dt / dx)**2
    
    # Inicialización de la solución
    u = np.zeros((Nt, Nx))
    informer.debug(u)
    
    # Condiciones iniciales
    u[0, :] = x * (1 - x)  # u(x, 0) = x(1 - x)
    
    # Primera iteración: derivada temporal cero
    u[1, :] = u[0, :]  # u_t(x, 0) = 0 implica que u[1, :] = u[0, :]
    
    # Aplicar condiciones de frontera
    u[:, 0] = 0  # u(0, t) = 0
    u[:, -1] = 0  # u(1, t) = 0
    
    # Iteraciones en el tiempo
    for n in range(1, Nt-1):
        for i in range(1, Nx-1):
            u[n+1, i] = (2 * u[n, i] - u[n-1, i] +
                         r * (u[n, i+1] - 2 * u[n, i] + u[n, i-1]) +
                         dt**2 * (1 - x[i]**2))
    
    return u, x, t

def jacobi(u, Nx, Ny, tol=1e-6, max_iter=10000):
    
    for iteracion in range(max_iter):
        u_old = np.copy(u)
        
        # Iterar sobre los puntos internos.
        for i in range(1, Nx-1):
            for j in range(1, Ny-1):
                # Esquema para la ecuacion de Laplace.
                u[i, j] = 0.25 * (u[i+1, j] + u[i-1, j] + u[i, j+1] + u[i, j-1])
        
        # Criterio de convergencia
        error = np.max(np.abs(u - u_old))
        if error < tol:
            informer.info(f"Convergencia alcanzada después de {iteracion} iteraciones.")
            return u, iteracion
        
    informer.info(f"No se alcanzó la convergencia después de {max_iter} iteraciones.")
    return u, max_iter


def gauss_seidel(u, Nx, Ny, tol=1e-6, max_iter=10000):
    """
    Método de Gauss-Seidel para resolver el sistema de ecuaciones discretizado
    de la ecuación de Laplace.
    
    Args:
    - u: Matriz con las condiciones iniciales de temperatura.
    - Nx: Número de puntos en la dirección x.
    - Ny: Número de puntos en la dirección y.
    - tol: Tolerancia para la convergencia.
    - max_iter: Máximo número de iteraciones.
    
    Returns:
    - u: Matriz con las soluciones aproximadas.
    - iteraciones: Número de iteraciones realizadas.
    """
    for iteracion in range(max_iter):
        max_error = 0.0
        
        # Iterar sobre los puntos internos
        for i in range(1, Nx-1):
            for j in range(1, Ny-1):
                # Guardar el valor anterior
                u_old = u[i, j]
                
                # Esquema para la ecuación de Laplace
                u[i, j] = 0.25 * (u[i+1, j] + u[i-1, j] + u[i, j+1] + u[i, j-1])
                
                # Calcular el error máximo
                max_error = max(max_error, abs(u[i, j] - u_old))
        
        
        # Criterio de convergencia
        if max_error < tol:
            informer.info(f"Convergencia alcanzada después de {iteracion} iteraciones.")
            return u, iteracion
        
    informer.info(f"No se alcanzó la convergencia después de {max_iter} iteraciones.")
    return u, max_iter


def gauss_seidel_matriz(A, b, tol=1e-6, max_iter=10000):
    """
    Método de Gauss-Seidel para resolver un sistema lineal Ax = b.
    
    Args:
    - A: Matriz de coeficientes.
    - b: Vector de términos independientes.
    - tol: Tolerancia para la convergencia.
    - max_iter: Máximo número de iteraciones.
    
    Returns:
    - x: Vector solución.
    - iteraciones: Número de iteraciones realizadas.
    """
    n = len(b)
    x = np.zeros_like(b, dtype=np.double)  # Vector inicial de solución
    
    for iteracion in range(max_iter):
        x_old = np.copy(x)
        
        # Iterar sobre cada ecuación del sistema
        for i in range(n):
            sigma = 0
            for j in range(n):
                if i != j:
                    sigma += A[i, j] * x[j]
            
            # Actualizar la solución usando los valores más recientes
            x[i] = (b[i] - sigma) / A[i, i]
        
        # Criterio de convergencia
        error = np.linalg.norm(x - x_old, ord=np.inf)
        if error < tol:
            informer.info(f"Convergencia alcanzada después de {iteracion} iteraciones.")
            return x, iteracion
    
    informer.info(f"No se alcanzó la convergencia después de {max_iter} iteraciones.")
    return x, max_iter


def gauss_seidel_sor(u, Nx, Ny, omega=1.5, tol=1e-6, max_iter=10000):
    
    for iteracion in range(max_iter):
        max_error = 0.0
        
        # Iterar sobre los puntos internos
        for i in range(1, Nx-1):
            for j in range(1, Ny-1):
                # Guardar el valor anterior
                u_old = u[i, j]
                
                # Esquema para la ecuación de Laplace (Gauss-Seidel + Sobrerrelajación)
                u_new = 0.25 * (u[i+1, j] + u[i-1, j] + u[i, j+1] + u[i, j-1])
                
                # Actualización con Sobrerrelajación
                u[i, j] = (1 - omega) * u_old + omega * u_new
                
                # Calcular el error máximo
                max_error = max(max_error, abs(u[i, j] - u_old))
        
        # Criterio de convergencia
        if max_error < tol:
            informer.info(f"Convergencia alcanzada después de {iteracion} iteraciones.")
            return u, iteracion
    
    informer.info(f"No se alcanzó la convergencia después de {max_iter} iteraciones.")
    return u, max_iter
\end{minted}


\subsubsection{parsers.py}
\begin{minted}[fontsize={\fontsize{5.5}{6.5}\selectfont}, breaklines]{python}
# /e/src/lib/parsers.py

import argparse
from .logger import (
    dict_log_level,
    dict_level_log
)
from ..core._typing import Any

def define_parser(mensaje_descripcion: str = "Este script procesa datos para MNA.") -> Any:

    parser = argparse.ArgumentParser(description=mensaje_descripcion)

    parser.add_argument(
        "-vsy", "--verbosity",
        type=int,
        choices=[level for level in dict_log_level.values()],
        default='INFO',
        help=f"Nivel de verbosidad {list(dict_log_level.items())}"
    )

    parser.add_argument(
        "-sp", "--show_plots", 
        action="store_true",
        help="Muestra los plots del script."
    )

    parser.add_argument(
        "-pl", "--parallel", 
        action="store_true",
        help="Hace los calculos (los que procedan) en paralelo."
    )
    
    return parser
\end{minted}


\subsubsection{placeholders.py}
\begin{minted}[fontsize={\fontsize{5.5}{6.5}\selectfont}, breaklines]{python}
# /e/src/lib/placeholders.py

from ..core._typing import Optional
from ..core._abstractas import Parametros

__all__ = [
    'ParametrosFisicos',
    'NpuntosDireccion',
    'ParametrosGeometricos',
    'ParametrosComputacionales'
]

class ParametrosFisicos(Parametros):
    pass

class NpuntosDireccion(Parametros):
    pass

class ParametrosGeometricos(Parametros):
    pass

class ParametrosComputacionales(Parametros):
    pass
\end{minted}

