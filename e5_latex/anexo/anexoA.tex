\chapter{main.py}\label{apendice:a}

\begin{minted}[fontsize={\fontsize{5.5}{6.5}\selectfont}, breaklines]{python}

# .

import numpy as np
import matplotlib.pyplot as plt
from scipy.linalg import solve
import pandas as pd

# Definimos funciones para obtener las matrices de rigidez y el vector de carga para elementos lineales y cuadráticos
def stiffness_matrix_linear(n_elements):
    """Genera la matriz de rigidez para elementos finitos lineales con n elementos"""
    n_nodes = n_elements + 1
    K = np.zeros((n_nodes, n_nodes))
    h = 1.0 / n_elements
    
    # Ensamblar la matriz de rigidez
    for i in range(n_elements):
        K[i, i] += 1 / h
        K[i, i + 1] += -1 / h
        K[i + 1, i] += -1 / h
        K[i + 1, i + 1] += 1 / h
    
    return K

def load_vector_linear(n_elements):
    """Genera el vector de carga para elementos finitos lineales con n elementos"""
    n_nodes = n_elements + 1
    f = np.zeros(n_nodes)
    h = 1.0 / n_elements
    
    # Ensamblar el vector de carga
    for i in range(n_elements):
        x_i = i * h
        x_i1 = (i + 1) * h
        f[i] += h / 2 * (x_i + h / 2)
        f[i + 1] += h / 2 * (x_i1 + h / 2)
    
    return f

def apply_boundary_conditions(K, f):
    """Aplica las condiciones de contorno u(0)=0 y du/dx(1)=0"""
    K[0, :] = 0
    K[0, 0] = 1
    f[0] = 0
    
    # La condición du/dx(1) = 0 ya está implícita en el problema (sin flujo)
    return K, f

# Matriz de rigidez para funciones cuadráticas
def stiffness_matrix_quadratic(n_elements):
    """Genera la matriz de rigidez para elementos finitos cuadráticos con n elementos"""
    n_nodes = 2 * n_elements + 1  # 3 nodos por elemento menos 1
    K = np.zeros((n_nodes, n_nodes))
    h = 1.0 / n_elements

    # Ensamblar la matriz de rigidez
    for i in range(n_elements):
        # Indices de los nodos de cada elemento
        idx = [2 * i, 2 * i + 1, 2 * i + 2]
        Ke = np.array([
            [7, -8, 1],
            [-8, 16, -8],
            [1, -8, 7]
        ]) * (1 / (3 * h))

        # Agregar el Ke a la matriz global
        for a in range(3):
            for b in range(3):
                K[idx[a], idx[b]] += Ke[a, b]

    return K

def load_vector_quadratic(n_elements):
    """Genera el vector de carga para elementos finitos cuadráticos con n elementos"""
    n_nodes = 2 * n_elements + 1  # 3 nodos por elemento menos 1
    f = np.zeros(n_nodes)
    h = 1.0 / n_elements

    # Ensamblar el vector de carga
    for i in range(n_elements):
        # Indices de los nodos de cada elemento
        idx = [2 * i, 2 * i + 1, 2 * i + 2]
        fe = np.array([1, 4, 1]) * (h / 6)  # Vector de carga para el término fuente lineal en x

        # Agregar el fe al vector global
        for a in range(3):
            f[idx[a]] += fe[a] * (2 * i + a) * h / 2  # Peso con x medio en cada subintervalo

    return f

# Resolver para el caso (a) - 4 elementos finitos, interpolación lineal
n_elements = 4
K = stiffness_matrix_linear(n_elements)
f = load_vector_linear(n_elements)
K_bc, f_bc = apply_boundary_conditions(K, f)

# Resolución del sistema
u = solve(K_bc, f_bc)

# Graficamos la solución obtenida
x = np.linspace(0, 1, n_elements + 1)

# Guardamos la solución para la comparación posterior
solutions = {"4_elements_linear": u}

# Resolver para el caso (b) - 8 elementos finitos, interpolación lineal
n_elements = 8
K = stiffness_matrix_linear(n_elements)
f = load_vector_linear(n_elements)
K_bc, f_bc = apply_boundary_conditions(K, f)

# Resolución del sistema
u_8_elements = solve(K_bc, f_bc)

# Graficamos la solución obtenida
x_8_elements = np.linspace(0, 1, n_elements + 1)

# Guardamos la solución para la comparación posterior
solutions["8_elements_linear"] = u_8_elements

# Resolver para el caso (c) - 2 elementos finitos, interpolación cuadrática
n_elements = 2
K_quad = stiffness_matrix_quadratic(n_elements)
f_quad = load_vector_quadratic(n_elements)
K_quad_bc, f_quad_bc = apply_boundary_conditions(K_quad, f_quad)

# Resolución del sistema
u_quad_2_elements = solve(K_quad_bc, f_quad_bc)

# Graficamos la solución obtenida
x_quad_2_elements = np.linspace(0, 1, 2 * n_elements + 1)

# Guardamos la solución para la comparación posterior
solutions["2_elements_quadratic"] = u_quad_2_elements

# Resolver para el caso (d) - 4 elementos finitos, interpolación cuadrática
n_elements = 4
K_quad = stiffness_matrix_quadratic(n_elements)
f_quad = load_vector_quadratic(n_elements)
K_quad_bc, f_quad_bc = apply_boundary_conditions(K_quad, f_quad)

# Resolución del sistema
u_quad_4_elements = solve(K_quad_bc, f_quad_bc)

# Graficamos la solución obtenida
x_quad_4_elements = np.linspace(0, 1, 2 * n_elements + 1)

# Guardamos la solución para la comparación posterior
solutions["4_elements_quadratic"] = u_quad_4_elements

# Definir la solución analítica de la ecuación diferencial
def analytical_solution(x):
    """Solución analítica de la ecuación diferencial"""
    return -x**2 / 2 + 0.5 * (x + np.sin(x)) 

# Definir la norma L2 para evaluar el error
def l2_error(u_num, u_analytical, x):
    """Calcula el error en norma L2 entre la solución numérica y la analítica"""
    error = np.sqrt(np.sum((u_num - u_analytical)**2 * np.diff(x, append=x[-1])))
    return error

# Crear el dominio fino para la solución analítica
x_fine = np.linspace(0, 1, 1000)
u_analytical = analytical_solution(x_fine)

# Graficar las soluciones obtenidas junto con la solución analítica
plt.figure(figsize=(10, 6))

# Solución analítica
plt.plot(x_fine, u_analytical, label="Solución Analítica", linestyle='--', color='black')

# Solución numérica con 4 elementos lineales
plt.plot(x, solutions["4_elements_linear"], label="4 elementos, lineal", marker='o')

# Solución numérica con 8 elementos lineales
plt.plot(x_8_elements, solutions["8_elements_linear"], label="8 elementos, lineal", marker='x')

# Solución numérica con 2 elementos cuadráticos
plt.plot(x_quad_2_elements, solutions["2_elements_quadratic"], label="2 elementos, cuadrático", marker='s')

# Solución numérica con 4 elementos cuadráticos
plt.plot(x_quad_4_elements, solutions["4_elements_quadratic"], label="4 elementos, cuadrático", marker='d')

plt.title("Comparación de soluciones numéricas y analítica")
plt.xlabel("x")
plt.ylabel("u(x)")
plt.legend()
plt.grid(True)
plt.show()

# Calcular errores en norma L2 para las diferentes soluciones numéricas
u_analytical_4 = analytical_solution(x)
u_analytical_8 = analytical_solution(x_8_elements)
u_analytical_quad_2 = analytical_solution(x_quad_2_elements)
u_analytical_quad_4 = analytical_solution(x_quad_4_elements)


# Calcular errores
errors = {
    "4 elementos lineal": l2_error(solutions["4_elements_linear"], u_analytical_4, x),
    "8 elementos lineal": l2_error(solutions["8_elements_linear"], u_analytical_8, x_8_elements),
    "2 elementos cuadrático": l2_error(solutions["2_elements_quadratic"], u_analytical_quad_2, x_quad_2_elements),
    "4 elementos cuadrático": l2_error(solutions["4_elements_quadratic"], u_analytical_quad_4, x_quad_4_elements),
}

# Mostrar los errores calculados
error_df = pd.DataFrame(list(errors.items()), columns=["Método", "Error L2"])

# Graficar los errores para un análisis visual de la convergencia
plt.figure(figsize=(8, 5))
methods = list(errors.keys())
error_values = list(errors.values())
plt.bar(methods, error_values, color=['blue', 'orange', 'green', 'red'])
plt.title("Errores L2 de los diferentes métodos")
plt.ylabel("Error L2")
plt.xticks(rotation=45, ha='right')
plt.grid(True)
plt.show()

# Función para realizar el cálculo de la solución numérica y el error para diferentes números de elementos
def compute_error_convergence(n_elements, interpolation_type="linear"):
    if interpolation_type == "linear":
        # Para elementos lineales
        K = stiffness_matrix_linear(n_elements)
        f = load_vector_linear(n_elements)
        K_bc, f_bc = apply_boundary_conditions(K, f)
        u_num = solve(K_bc, f_bc)
        x = np.linspace(0, 1, n_elements + 1)
    elif interpolation_type == "quadratic":
        # Para elementos cuadráticos
        K = stiffness_matrix_quadratic(n_elements)
        f = load_vector_quadratic(n_elements)
        K_bc, f_bc = apply_boundary_conditions(K, f)
        u_num = solve(K_bc, f_bc)
        x = np.linspace(0, 1, 2 * n_elements + 1)
    
    # Solución analítica en los puntos correspondientes
    u_analytical = analytical_solution(x)
    
    # Cálculo del error L2
    error = l2_error(u_num, u_analytical, x)
    
    return error

# Número de elementos a probar para el estudio de convergencia
element_counts = [2, 4, 8, 16, 32]
errors_linear = []
errors_quadratic = []

# Calcular errores para interpolaciones lineales y cuadráticas
for n_elements in element_counts:
    error_linear = compute_error_convergence(n_elements, interpolation_type="linear")
    error_quadratic = compute_error_convergence(n_elements, interpolation_type="quadratic")
    errors_linear.append(error_linear)
    errors_quadratic.append(error_quadratic)

# Graficar la convergencia en un gráfico log-log
plt.figure(figsize=(8, 6))
plt.loglog(element_counts, errors_linear, label="Interpolación lineal", marker='o')
plt.loglog(element_counts, errors_quadratic, label="Interpolación cuadrática", marker='s')
plt.xlabel("Número de elementos")
plt.ylabel("Error L2")
plt.title("Estudio de convergencia")
plt.legend()
plt.grid(True, which="both", ls="--")
plt.show()




\end{minted}